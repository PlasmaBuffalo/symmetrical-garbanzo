\documentclass{article}  % determines the type of document we're creating.

\usepackage{amsmath , amssymb , amsthm}

\newcommand{\ds}{\displaystyle}
\newcommand{\e}{\varepsilon}

%  The percent sign is the character used for making comments.
% Everything after the percent sign on a single line will be ignored by the LaTeX software.

% Everything above is preamble.  Loading packages into the software and assigning new commands.

% This entire document is a program that runs in LaTeX to create a nice typeset document (typically a .pdf).  


\begin{document}  
% Everything between this command and the \end{document} command below is contained in the document environment.  Within this environment
% we can also enter environments like enumerate and itemize below.  


\pagestyle{empty}
\centerline{\Large LaTeX Example}

\bigskip

Welcome to \LaTeX (pronounced ``Lay-Tech''), the most popular mathematical typesetting language in the world.                         This PDF document \texttt{latexsample2.pdf} was created by running \LaTeX{} on a program contained in the file \texttt{latexsample2.tex}, whichF
was written with                                  the purpose of creating this document.  
This document contains a few random examples that might help you when                                       typing.  
The most important character is \verb#\#, typically located on the upper right of your keyboard.  Every \LaTeX{} command starts with this character.  I made this backslash by typing \verb^\verb#\#^ to use the ``verbatim mode''.  
 I could have also used \verb#\textbackslash# to make \textbackslash.  
 Notice the nice shape of my quotation marks in the  sentences above as opposed to this "quote" where I wrongly used the double quotation character.

% Notice the extra space above.  Doesn't make a difference in the document.  
% LaTeX takes care of spacing for you.  Also, you only get a new paragraph if you  have an empty line between paragraphs here.



Notice also that many commands take inputs, which go between the curly brace grouping symbols \{ and \}.  There are many  resources for learning about \LaTeX, including Google.

\section{Lists}

\begin{enumerate}  % enumerate is an environment.  So is itemize below.
\item This is a list.

%Notice how none of this space matters.



\item Here is the second item.
\end{enumerate}

There are other ways of making lists.

\begin{itemize}
\item Sometimes you'd like a list \dots
\item \dots without numbers.
\end{itemize}

\section{Math}

The first rule (see the next document) is that all mathematics must be put in a mathematical environment.  

Inline mathematical symbols like $e$ or $\pi$ are always enclosed within dollar signs \$.

I like this integral: $ \int_0^1 \frac{dx}{\sqrt x}.$

If you wanted to center that text, you could put it in the \texttt{center} environment:

\begin{center}
$\sum_{n = 1}^{\infty} \frac{1}{n^2} = \frac{\pi^2}{6}.$
\end{center}

You can let the math be larger with $\backslash$\texttt{displaystyle}:


$\displaystyle \hbox{Area}\left( \bigcup_{n = 1}^{\infty} \frac{1}{n}\times \frac{1}{n}\hbox{ square}\right) = \frac{\pi^2}{6},$
or you could do both by using, instead of the inline \$ \dots \$ environment, the display math environment with $\backslash[$ and $\backslash]$:

\[
\sum_{i=1}^n i^2 = \frac{n(n+1)(2n+1)}{6}.
\]

Or we could use the \texttt{equation} environment to make a numbered equation:

\begin{equation}
 \int_0^1 \frac{dx}{\sqrt x}.
\end{equation}

Wait -- is that  an equation?

The \texttt{array} environment is great for people who love matrices, such as
\[
M=\left(
\begin{array}{ccc}
1 & 2&3 \\
4&5&6 \\
7&8&9
\end{array}
\right)
\]

\section{Spacing}
One can skip space vertically or horizontally with \texttt{$\backslash$bigskip, $\backslash$medskip, $\backslash$smallskip}
or $\backslash$vspace\{1.3in\} and $\backslash$hspace\{0.2cm\}.  Remove paragraph indentations with $\backslash$noindent.  Another useful horizontal spacer is \textbackslash qquad.

\vspace{0.3in}

\noindent
Look, I made a $0.3$ inch space!  In math mode, we can use spacers like \textbackslash , to make $\ds \int f(x) dx$ look better: $\ds \int f(x)\; dx$

\section{More Symbols}

Some useful logical symbols include $\forall$,  $\exists$, $\in$,  $\ni$, $\subseteq$, and $\supseteq$.

Convergent sequences look like this:

\[
\lim_{n \to \infty} a_n = L.
\]

There are several funky symbols you might want - they are included in the AMS Symbol package.  (You need to put  \verb#\usepackage{amssymb}# in the preamble.)

${\mathbb R}$ %\hspace{0.3in}
\qquad ${\mathbb Q}$ \qquad $\square$ \qquad$\blacksquare$ \qquad $\e$ \qquad $\emptyset$ \qquad $\mathcal P$ \qquad $\mathcal N$

\end{document}